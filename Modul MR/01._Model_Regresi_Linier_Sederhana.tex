\begin{sloppypar}
Model regresi linier sederhana merupakan suatu persamaan yang menggambarkan hubungan antara variabel bebas/predictor (X) dan satu variabel tak bebas/ response (Y). Tujuan dari analisis regresi linier sederhana adalah untuk memprediksi nilai variabel terikat (Y) jika nilai variabel-variabel bebas (X) diketahui, serta mengetahui arah hubungan antara variabel tak bebas dengan variabel-variabel bebas.
\\Asumsi-asumsi yang harus dipenuhi agar model dapat dinyatakan valid:
\begin{enumerate}
    \item[$\bullet$] $\beta_i$ untuk i = 0, 1 adalah parameter yang belum diketahui.
    \item[$\bullet$] X adalah variabel bebas/predictor (independen) yang diatur tanpa error.
    \item[$\bullet$] Y adalah variabel tak bebas/response (dependen) yang digunakan sebagai prediksi berdasarkan nilai variabel X.
    \item[$\bullet$] $\epsilon$ adalah komponen error random yang saling bebas dan mempunyai distribusi normal dengan rata-rata dan varian ($\sigma^2$) yang ditentukan berdasarkan nilai data variabel tak bebas.
    \item[$\bullet$] Hubungan variabel X dan Y adalah linier.
\end{enumerate}
\end{sloppypar}

\noindent Bentuk Umum Model Regresi Linier Sederhana:
\begin{equation}
    Y = \beta_{0} + \beta_{1}X + \epsilon
    \label{eq1}
\end{equation}
Bentuk Umum Estimasi Model Regresi Linier Sederhana: 
\begin{equation}
    \hat{Y} = \widehat{\beta_{0}} + \widehat{\beta_{1}}X
    \label{eq2}
\end{equation}
\begin{test}{
    Pada Tabel Coefficients, pada kolom Unstandardized B, dapat diperoleh : \\
    $\beta_0$ = 30.871, $\beta_1$ = 43.413 \\
    Maka persamaan regresi sebagai berikut : \\
    \begin{equation*}
        \hat{Y} = 30.87 + 43.413 X
    \end{equation*}
}
library(lmtest)
library(nortest)
library(readxl)
library(zoo)

data_Taiwan <- readxl::read_excel("D:\\FAR-FadhZ\\M.Excel\\Data_TaiwanAQI.xlsx")

View(data_Taiwan)
Y <- data_Taiwan$AQI
X <- data_Taiwan$CO_8hr

plot(Y, X, main='Scatter PLot Data', xlab='Rata-rata konsentrasi CO dalam 8 jam', ylab='Indeks kualitas air')

regresi <- lm(Y~X, data = data_Taiwan)
summary(regresi)
\end{test}

\begin{enumerate}
\def\labelenumi{\arabic{enumi}.}
\tightlist
\item Uji Asumsi

    \begin{enumerate}
    \def\labelenumii{(\alph{enumii})}
    \tightlist
    \item Uji Normalitas \\
    \begin{sloppypar}
    Uji normalitas bertujuan untuk menguji apakah dalam model regresi variabel dependen dan variabel independen keduanya mempunyai distribusi normal. Model regresi yang baik memiliki distribusi data yang normal. Ada dua cara untuk mendeteksi apakah residual berdistribusi normal atau tidak normal yaitu dengan analisis grafik (histogram dan \textit{probability plot}) dan uji \textit{Kolmogorov-Smirnov}. Dasar pengambilan keputusan uji \textit{Kolmogorov-Smirnov} adalah :
    \end{sloppypar}
        \begin{itemize}
        \item Jika nilai signifikansi $>$ 0.05 maka nilai residual berdistribusi normal
        \item Jika nilai signifikansi $<$ 0.05 maka nilai residual tidak berdistribusi normal
        \end{itemize}
        \begin{test}{
            \begin{enumerate}
            \item[-] Rumusan hipotesis \\
            \begin{fleqn}[\parindent]
                \begin{equation*}
                \begin{split}
                &H_0 : \text{(data) nilai residual berdistribusi normal} \\ 
                &H_1 : \text{(data) nilai residual tidak berdistribusi normal} \\
                \end{split}
                \end{equation*}
            \end{fleqn}
        
            \item[-] Taraf Signifikansi \\
            $\alpha = 0.05$
        
            \item[-] Statistik Uji \\
            Pada Tabel Tests of Normality, dapat diperoleh: \\
            Nilai sig K-S = 0.200
        
            \item[-] Daerah Kritis \\
            $H_0$ ditolak jika nilai sig K-S $< \alpha$
        
            \item[-] Keputusan \\
            $H_0$ gagal ditolak karena nilai signifikansi Kolmogorov-Smirnov $< \alpha$, yaitu $0.200 > 0.05$
        
            \item[-] Kesimpulan \\
            Pada taraf signifikansi 5\%, $H_0$ gagal ditolak sehingga residual data berdistribusi normal menurut uji Kolmogorov-Smirnov
        \end{enumerate}
        }
shapiro.test(regresi$residuals)
nortest::lillie.test(regresi$residuals)
plot(regresi, 2)
        \end{test}

    \item Uji Linieritas \\
    \begin{sloppypar}
    Uji liniaritas bertujuan untuk menguji apakah pada model regresi ditemukan adanya korelasi/hubungan yang linier antar variabel independen. Pada model regresi yang baik seharusnya antar variabel independen tidak terjadi kolerasi. Untuk mendeteksi ada tidaknya liniearitas dalam model regresi dapat dilihat dari nilai signifikan baris \textit{Deviation from Linearity} pada tabel ANOVA.
    \end{sloppypar}
        \begin{test}{
            \begin{enumerate}
            \item[-] Rumusan hipotesis \\
            \begin{fleqn}[\parindent]
                \begin{equation*}
                \begin{split}
                &H_0 : \text{Residual regresi memiliki pola data linier} \\ 
                &H_1 : \text{Residual regresi memiliki pola data tidak linier} \\
                \end{split}
                \end{equation*}
            \end{fleqn}
        
            \item[-] Taraf Signifikansi \\
            $\alpha = 0.05$
        
            \item[-] Statistik Uji \\
            Pada Tabel ANOVA baris Deviation from Linearity, dapat diperoleh: \\
            Nilai sig = 0.080
        
            \item[-] Daerah Kritis \\
            $H_0$ ditolak jika nilai sig $< \alpha$
        
            \item[-] Keputusan \\
            $H_0$ gagak ditolak karena nilai signifikansi Deviation from Linearity $< \alpha$, yaitu $0.080 > 0.05$
        
            \item[-] Kesimpulan \\
            Pada taraf signifikansi 5\%, $H_0$ gagal ditolak sehingga residual regresi memiliki pola data linier
        \end{enumerate}
        }
plot(regresi, 1)
        \end{test}

    \item Uji Homoskedastisitas \\
    \begin{sloppypar}
    Uji homoskedastisitas digunakan dalam menguji error atau galat dalam model statistik untuk melihat apakah varians atau keragaman dari error terpengaruh oleh faktor lain atau tidak. Asumsi divisualisasi pada grafik \textit{SRESID} by \textit{ZPRED Scatterplot}.
    \begin{itemize}
    \item Jika ada pola tertentu, seperti titik-titik yang ada membentuk pola tertentu yang teratur (bergelombang, melebar kemudian menyempit), maka mengindikasikan telah terjadi heteroskedastisitas.  
    \item Jika tidak ada pola yang jelas, serta titik-titik menyebar di atas dan di bawah angka 0 pada sumbu Y, maka mengindikasikan telah terjadi homoskedastisitas
    \end{itemize}
    \end{sloppypar}
        \begin{test}{
            \begin{enumerate}
            \item[-] Rumusan hipotesis \\
            \begin{fleqn}[\parindent]
                \begin{equation*}
                \begin{split}
                &H_0 : \text{Residual regresi homoskedastisitas} \\ 
                &H_1 : \text{Residual regresi heteroskedastisitas} \\
                \end{split}
                \end{equation*}
            \end{fleqn}
        
            \item[-] Taraf Signifikansi \\
            $\alpha = 0.05$
        
            \item[-] Statistik Uji \\
            Pada Tabel Coefficients dari regresi variable X dengan mutlak residual regresi variabel X dan Y, dapat diperoleh: \\
            Nilai sig = 0.000
        
            \item[-] Daerah Kritis \\
            $H_0$ ditolak jika nilai sig $< \alpha$
        
            \item[-] Keputusan \\
            $H_0$ ditolak karena nilai signifikansi $< \alpha$, yaitu $0.000 < 0.05$
        
            \item[-] Kesimpulan \\
            Pada taraf signifikansi 5\%, $H_0$ ditolak sehingga residual regresi heteroskedastisitas
        \end{enumerate}
        }
par(mfrow = c(2, 2))
plot(regresi, 3)
glejser_test <- lm(abs(regresi$residuals)~X, data = data_Taiwan)
summary(glejser_test)
        \end{test}
    
    \item Uji Autokorelasi \\
    \begin{sloppypar}
    Uji autokorelasi bertujuan menguji apakah dalam model regresi linear ada korelasi antara kesalahan pengganggu pada periode t dengan kesalahan pengganggu pada periode t-1 (sebelumnya). Uji autokorelasi dalam penelitian ini menggunakan uji Durbin Watson. Oleh karena itu, apabila asumsi autokorelasi terjadi pada sebuah model prediksi, maka nilai disturbance tidak lagi berpasangan secara bebas, melainkan berpasangan secara autokorelasi.
    \end{sloppypar}
    \begin{equation*}
        d = \frac{\sum_{i=2}^N (e_i - e_{i-1})^2}{\sum_{i=1}^N e_i^2}
    \end{equation*}
    \begin{test}{
        \begin{enumerate}
        \item[-] Rumusan hipotesis \\
        \begin{fleqn}[\parindent]
            \begin{equation*}
            \begin{split}
            &H_0 : \text{(data) tidak terdapat autokorelasi} \\ 
            &H_1 : \text{(data) terdapat autokorelasi} \\
            \end{split}
            \end{equation*}
        \end{fleqn}
    
        \item[-] Taraf Signifikansi \\
        $\alpha = 0.05$
    
        \item[-] Statistik Uji \\
        Pada Tabel Model Summary, dapat diperoleh : \\
        Nilai d = 0.878
    
        \item[-] Daerah Kritis \\
        Pengambilan keputusan ada/tidaknya autokorelasi disajikan pada :
        \begin{enumerate}
        \item[$\square$] $0 < d < dl$ (menolak $H_0$, autokorelasi positif)
        \item[$\square$] $dl < d < du$ (daerah ragu-ragu)
        \item[$\square$] $du < d < 4-du$ (menerima $H_0$, atau tidak ada autokorelasi)
        \item[$\square$] $4-du < d < 4-dl$ (daerah ragu-ragu)
        \item[$\square$] $4-dl < d < 4$ (menolak $H_0$, autokorelasi negatif)
        \end{enumerate}
        dengan, \\
        n (jumlah data) = 65, k (variabel) = 1, dl = 1.5670, du = 1.6294, 4-du = 2.3706, 4-dl = 2.4330
    
        \item[-] Keputusan \\
        $H_0$ ditolak karena 0 $<$ nilai Durbin-Watson $<$ dl, yaitu $0 < 0.878 < 1.5670$
    
        \item[-] Kesimpulan \\
        Pada taraf signifikansi 5\%, $H_0$ ditolak sehingga terdapat autokorelasi positif antar residual variabel X dan Y
    \end{enumerate}
    }
lmtest::dwtest(regresi)
    \end{test}

    \end{enumerate}

\item Uji Kecocokan Model \\
\begin{sloppypar}
Uji F ini bertujuan untuk menguji apakah variabel independen secara bersama-sama (simultan) mempengaruhi variabel dependen. Uji F dilakukan untuk melihat pengaruh dari seluruh variabel bebas secara bersama-sama terhadap variabel terikat. 
\end{sloppypar}
$$\begin{array}{llllll}
    \hline 
    \text{Komponen Regresi} & \text{SS} & \text{db} & \text{MS} & \text{Nilai F} \\
    \hline 
    \text{Regresi} & \text{JKR} = \sum_{i=1}^{n}(\hat{Y}_i - \bar{Y})^2 & 1 & \text{RKR} = \text{JKR} & \text{F}_\text{hitung} = \frac{\text{RKR}}{\text{RKS}} \\
    \text{Galat/Sisa} & \text{JKS} = \sum_{i=1}^{n}(Y_i - \hat{Y}_i)^2 & n-2 & \text{RKS} = \frac{\text{JKS}}{n-2} \\
    \text{Total} & \text{JKT} = \sum_{i=1}^{n}(Y_i - \bar{Y})^2 & n-1 \\
    \hline
\end{array}$$
\begin{table}[h]
    \begin{tabular}{p{4cm} p{2.5cm} p{6cm}}
        \toprule
\textbf{Kriteria} & \textbf{Keputusan} & \textbf{Artinya} \\
        \midrule
Jika nilai Sig F $>$ 0.05 atau nilai F Hitung $<$ F Tabel
& $H_0$ gagal ditolak 
& Variabel independen (X) secara simutan tidak berpengaruh signifikan terhadap variabel dependen (Y) \\  
Jika nilai Sig F $<$ 0.05 atau nilai F Hitung $>$ F Tabel
& $H_0$ ditolak       
& Variabel independen (X) secara simultan berpengaruh signifikan terhadap variabel dependen (Y) \\
        \bottomrule
    \end{tabular}
\end{table}
\begin{test}{
    \begin{enumerate}
    \item[-] Rumusan hipotesis \\
    \begin{fleqn}[\parindent]
        \begin{equation*}
        \begin{split}
        &H_0 : \beta = 0 \\ 
        &H_1 : \beta \neq 0 \\
        \end{split}
        \end{equation*}
    \end{fleqn}
    Interpretasi:
    \begin{enumerate}
    \item[$\square$] $H_0$ didefinisikan model regresi tidak cocok
    \item[$\square$] $H_1$ didefinisikan model regresi cocok
    \end{enumerate}

    \item[-] Taraf Signifikansi \\
    $\alpha = 0.05$

    \item[-] Statistik Uji \\
    Pada Tabel Model ANOVA, dapat diperoleh : \\
    F-hitung = 9.009, Nilai sig = 0.004 \\
    Berdasarkan tabel F didapat : \\
    F-tabel = $F_{\alpha, k, n-k-1}$ = $F_{0.05; 1; 63}$ = 3.99

    \item[-] Daerah Kritis \\
    $H_0$ ditolak jika $F_0 > F_{\alpha, k, n-k-1}$ atau nilai sig $< \alpha$

    \item[-] Keputusan \\
    $H_0$ ditolak karena $F_0 > F_{\alpha, k, n-k-1}$, yaitu $9.009 > 3.99$

    \item[-] Kesimpulan \\
    Pada taraf signifikansi 5\%, $H_0$ ditolak sehingga model regresi cocok memiliki parameter $\beta$ yang tak nol di paling tidak salah satu parameter
\end{enumerate}
}
summary(regresi)
\end{test}

\item Uji Signifikansi Parameter \\
\begin{sloppypar}
Uji t berguna untuk mengetahui apakah model regresi yang terbentuk variable-variabel bebasnya (X) secara parsial berpengaruh signifikan terhadap variable terikat Y.
\end{sloppypar}
$$\begin{array}{lll}
    \hline 
    \text{Parameter }\beta & \text {Nilai t} & \text {Standar Error} \\
    \hline 
    0 & t_0 = \frac{b_0 - \beta_{00}}{s_{b_0}} & s_{b_0} = \frac{s_e}{\sqrt{S_{xx}}} \\
    1 & t_1 = \frac{b_1 - \beta_{10}}{s_{b_1}} & s_{b_1} = s_e \sqrt{\frac{1}{n} + \frac{\overline{X^2}}{S_{xx}}} \\
    \hline
\end{array}$$
\begin{table}[h!]
    \begin{tabular}{p{4cm} p{2.5cm} p{6cm}}
        \toprule
\textbf{Kriteria} & \textbf{Keputusan} & \textbf{Artinya} \\ 
\midrule
Jika nilai Sig t $>$ 0.05 atau nilai t Hitung $<$ t Tabel
& $H_0$ gagal ditolak 
& Variabel independen (X) secara parsial tidak berpengaruh signifikan terhadap variabel dependen (Y) \\     
Jika nilai Sig t $<$ 0.05 atau nilai t Hitung $>$ t Tabel
& $H_0$ ditolak       
& Variabel independen (X) secara partial berpengaruh signifikan terhadap variabel dependen (Y) \\
        \bottomrule
    \end{tabular}
\end{table}
\begin{test}{
    \begin{enumerate}
    \item[-] Rumusan hipotesis \\
    \begin{fleqn}[\parindent]
        \begin{equation*}
        \begin{split}
        &H_0 : \beta_i = 0 \text{, paling sedikit 1 untuk i = 0, 1} \\ 
        &H_1 : \beta_i \neq 0 \text{, paling sedikit 1 untuk i = 0, 1}\\
        \end{split}
        \end{equation*}
    \end{fleqn}
    Interpretasi:
    \begin{enumerate}
    \item[$\square$] $H_0$ didefinisikan koefisien parameter tidak sesuai
    \item[$\square$] $H_1$ didefinisikan koefisien parameter sesuai
    \end{enumerate}

    \item[-] Taraf Signifikansi \\
    $\alpha = 0.05$

    \item[-] Statistik Uji \\
    Pada Tabel Model ANOVA, dapat diperoleh : \\
    t-hitung 0 = 4.493, t-hitung 1 = 3.002, Nilai sig 0 = 0.000, Nilai sig 1 = 0.004 \\
    Berdasarkan tabel t didapat : \\
    t-tabel = $t_{\alpha/2, n-k-1}$ = $t_{0.025; 63}$ = 1.9983

    \item[-] Daerah Kritis \\
    $H_0$ ditolak jika $|t_0| > t_{\alpha, n-k-1}$ atau nilai sig $< \alpha$

    \item[-] Keputusan \\
    $$\begin{array}{lllll}
        \hline 
        \text{Variabel} & \text{Nilai t} & \text{Sig} & \text{Keputusan}  \\
        \hline 
        t_0 & 4.493 & 0.000 & H_0 \text{ ditolak} \\
        t_1 & 3.002 & 0.004 & H_0 \text{ ditolak} \\
        \hline
    \end{array}$$

    \item[-] Kesimpulan \\
    Pada taraf signifikansi 5\%, $H_0$ ditolak untuk koefisien parameter $\beta_0$ dan $\beta_1$ yang tak nol sehingga variabel X berpengaruh signifikan terhadap variabel Y
\end{enumerate}
}
summary(regresi)
\end{test}

\item Koefisien Korelasi \\
\begin{sloppypar}
Koefisien korelasi ganda digunakan untuk mengetahui seberapa besar korelasi yang terjadi antara variable-variabel X secara serentak/simultan dengan variabel Y. Nilai r berkisar antara -1 dan 1. $$-1 \leq r \leq 1$$ Nilai ketepatan model regresi dengan koefisien korelasi ganda : $$r = \sqrt{R} = \frac{S_{xy}}{\sqrt{S_{xx} S_{yy}}}$$
\end{sloppypar}
\begin{itemize}
    \item Jika nilai koefisien korelasi ganda mendekati satu \(+1 atau -1\) berarti variabel-variabel independen mempunyai hubungan kuat dengan variabel dependen. Apabila mendekati +1, dipastikan variabel independen hubungan kuat positif terjadi, begitu pula sebaliknya  
    \item Jika nilai koefisien korelasi ganda mendekati nol berarti variabel independen mempunyai hubungan lemat atau tidak mempunyai hubungan dengan variabel variabel dependen.    
\end{itemize}
\begin{test}{
    Pada Tabel Model Summary, dapat diperoleh : \\
    r = 0.354
}
summary(regresi)
\end{test}

\item Koefisien Determinasi \\
\begin{sloppypar}
Koefisien determinasi digunakan untuk mengukur seberapa jauh kemampuan variabel independen dalam menerangkan variasi perubahan variabel dependen. Nilai koefisien determinasi adalah 0 dan 1. $$0 \leq R^2 \leq 1$$ Persamaan tersebut menyatakan jumlah variabilitas dalam data dalam model dan kontribusi variabel bebas x terhadap variabel respon y. Nilai ketepatan model regresi dengan koefisien determinasi : $$R^2 = \frac{\text{JKR}}{\text{JKT}}$$
\end{sloppypar}
\begin{itemize}
    \item Jika nilai koefisien determinasi mendekati satu berarti variabel-variabel independen memberikan hampir semua informasi yang dibutuhkan untuk untuk memprediksi variasi variabel dependen.
    \item Jika nilai koefisien determinasi mendekati nol berarti variabel independen memberikan sedikit informasi yang dibutuhkan untuk memprediksi variasi variabel dependen
\end{itemize}
\begin{test}{
    Pada Tabel Model Summary, dapat diperoleh : \\
    $R^2$ = 0.125
}
summary(regresi)
\end{test}

\end{enumerate}
