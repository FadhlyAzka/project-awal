\begin{sloppypar}
Data multinomial merupakan data yang memiliki variabel hasil yang bersifat nominal atau polikotomi dengan lebih dari dua level. Model regresi logistik multinomial merupakan suatu metode analisis data yang digunakan untuk mencari hubungan antara variabel terikat (Y) yang bersifat polikotomus atau multinomial, yaitu (1, ..., J), dengan beberapa variabel bebas/predictor ($X_1, X_2, ..., X_n$). 
Tujuan dari analisis regresi logistik multinomial adalah untuk menganalisis hubungan antara satu variabel terikat (Y) dan beberapa variabel bebas/predictor ($X_1, X_2, ..., X_n$), dengan variabel terikat (Y) berupa data kualitatif yang mempunyai kategori lebih dari 2.
\\ Model regresi logistik diperoleh dari fungsi logistik dengan didefinisikan sebagai
\begin{equation}
\pi_j(x) = \frac{e^{(\beta_0\ +\ \beta_{1j} x_1\ +\ \cdots\ +\ \beta_{pj} x_p)}}{1 + \sum_{j=1}^{J-1} e^{(\beta_0\ +\ \beta_{1j} x_1\ +\ \cdots\ +\ \beta_{pj} x_p)}}
\end{equation}
untuk j = (1, ..., J-1) dan j = J [level terakhir] ialah
\begin{equation}
\pi_J(x) = \frac{1}{1 + \sum_{j=1}^{J-1} e^{(\beta_0\ +\ \beta_{1j} x_1\ +\ \cdots\ +\ \beta_{pj} x_p)}}
\end{equation}
\end{sloppypar}
\begin{test}{
    Pada Tabel Variables in the Equation, dapat diperoleh : \\
    $\beta_{10}$ = 1.120, $\beta_{11}$ = -0.060, $\beta_{12}$ = -0.058, $\beta_{13}$ = 0, $\beta_{14}$ = -1.122, $\beta_{15}$ = 0, $\beta_{16}$ = -1.018, $\beta_{17}$ = 0 \\
    $\beta_{20}$ = 0.802, $\beta_{21}$ = -0.052, $\beta_{22}$ = -0.164, $\beta_{23}$ = 0, $\beta_{24}$ = -1.083, $\beta_{25}$ = 0, $\beta_{26}$ = 0.153, $\beta_{27}$ = 0 \\
    Maka persamaan regresi sebagai berikut : \\
    \begin{equation*}
    \begin{aligned}
        \pi_1(x) &= \frac{e^{1.120 - 0.060 x_{11} - 0.058 x_{12} - 1.122 x_{14} - 1.018 x_{16}}}{\splitfrac{1 + e^{1.120 - 0.060 x_{11} - 0.058 x_{12} - 1.122 x_{14} - 1.018 x_{16}}}{+ e^{0.802 - 0.052 x_{21} - 0.164 x_{22} - 1.083 x_{24} + 0.153 x_{26}}}} \\
        \pi_2(x) &= \frac{e^{0.802 - 0.052 x_{21} - 0.164 x_{22} - 1.083 x_{24} + 0.153 x_{26}}}{\splitfrac{1 + e^{1.120 - 0.060 x_{11} - 0.058 x_{12} - 1.122 x_{14} - 1.018 x_{16}}}{+ e^{0.802 - 0.052 x_{21} - 0.164 x_{22} - 1.083 x_{24} + 0.153 x_{26}}}} \\
        \pi_3(x) &= \frac{1}{\splitfrac{1 + e^{1.120 - 0.060 x_{11} - 0.058 x_{12} - 1.122 x_{14} - 1.018 x_{16}}}{+ e^{0.802 - 0.052 x_{21} - 0.164 x_{22} - 1.083 x_{24} + 0.153 x_{26}}}}
    \end{aligned}
    \end{equation*}
}
\end{test}

\begin{enumerate}
    \def\labelenumi{\arabic{enumi}.}
    \tightlist
    \item Uji Simultan
    \begin{test}{
    \begin{enumerate}
        \item[-] Rumusan Hipotesis 
        \begin{fleqn}[\parindent]
            \begin{align*}
            H_0 &: \beta_{1j} = \beta_{2j} = \beta_{3j} = \beta_{4j} = \beta_{5j} = \beta_{6j} = \beta_{7j} = 0, \\  &\quad \text{untuk semua j = 1, 2, 3}\\ 
            H_1 &: \beta_{ij} \neq 0 \text{, paling sedikit 1 untuk i = 1, 2, 3, 4, 5, 6, 7  } \\ &\quad \text{dan semua j = 1, 2, 3}
            \end{align*}
        \end{fleqn}
        Interpretasi:
        \begin{enumerate}
        \item[$\square$] $H_0$ didefinisikan secara simultan variabel bebas/prediktor tidak berpengaruh terhadap variabel respon
        \item[$\square$] $H_1$ didefinisikan secara simultan variabel bebas/prediktor berpengaruh terhadap variabel respon
        \end{enumerate}
    
        \item[-] Taraf Signifikansi \\
        $\alpha = 0.05$
    
        \item[-] Statistik Uji \\
        Pada Tabel Model Fitting Information, dapat diperoleh : \\
        G = 35.041, Nilai sig = 0.000 \\
        Berdasarkan tabel F didapat : \\
        $\chi$-tabel = $\chi_{\alpha, k}^2$ = $\chi_{0.05; 8}$ = 15.507
    
        \item[-] Daerah Kritis \\
        $H_0$ ditolak jika $G > \chi_{\alpha, k}^2$ atau nilai sig $< \alpha$
    
        \item[-] Keputusan \\
        $H_0$ ditolak karena $G > \chi_{\alpha, k}^2$, yaitu $35.041 > 15.507$
    
        \item[-] Kesimpulan \\
        Pada taraf signifikansi 5\%, $H_0$ ditolak sehingga secara simultan variabel bebas/prediktor berpengaruh terhadap variabel respon. 
    \end{enumerate}
    }
    \end{test}
    
    \item Uji Parsial
    \begin{test}{
    \begin{enumerate}
        \item[-] Rumusan Hipotesis \\
        \begin{fleqn}[\parindent]
            \begin{equation*}
            \begin{split}
            &H_0 : \beta_{ij} = 0 \text{, untuk setiap i = 1, 2, 3, 4, 5, 6, 7 dan semua j = 1, 2, 3} \\ 
            &H_1 : \beta_{ij} \neq 0 \text{, untuk setiap i = 1, 2, 3, 4, 5, 6, 7 dan semua j = 1, 2, 3} \\
            \end{split}
            \end{equation*}
        \end{fleqn}
        Interpretasi:
        \begin{enumerate}
        \item[$\square$] $H_0$ didefinisikan variabel bebas/prediktor tidak signifikan terhadap model
        \item[$\square$] $H_1$ didefinisikan variabel bebas/prediktor signifikan terhadap model
        \end{enumerate}
    
        \item[-] Taraf Signifikansi \\
        $\alpha = 0.05$
    
        \item[-] Statistik Uji \\
        Pada Tabel Variables in the Equation, dapat diperoleh : \\
        $W_{11}$ = 0.014, $W_{12}$ = 0.015, $W_{14}$ = 11.917, $W_{16}$ = 9.792, $W_{21}$ = 0.012, $W_{22}$ = 0.139, $W_{24}$ = 12.464, $W_{26}$ = 0.251,  
        \\ Nilai sig 11 = 0.906, Nilai sig 12 = 0.901, Nilai sig 14 = 0.001, Nilai sig 16 = 0.002, Nilai sig 21 = 0.912, Nilai sig 22 = 0.710, Nilai sig 24 = 0.000, Nilai sig 16 = 0.616 \\
        Berdasarkan tabel F didapat : \\
        $\chi$-tabel = $\chi_{\alpha, k}^2$ = $\chi_{0.05; 1}$ = 3.841
    
        \item[-] Daerah Kritis \\
        $H_0$ ditolak jika $W_j > \chi_{\alpha, k}^2$ atau $|W_j^*| > Z_{\frac{\alpha}{2}}$ atau nilai sig $< \alpha$
    
        \item[-] Keputusan \\
        \begin{tabular}{lllll}
            \hline 
            \text{Kategori} & \text{Variabel} & \text{Nilai Wald} & \text{Sig} & \text{Keputusan}  \\
            \hline 
            \multirow{4}{*}{Mediasi} & $W_{11}$ & 0.014 & 0.906 & $H_0 \text{ gagal ditolak}$ \\
            & $W_{12}$ & 0.015 & 0.901 & $H_0 \text{ gagal ditolak}$ \\
            & $W_{14}$ & 11.917 & 0.001 & $H_0 \text{ ditolak}$ \\
            & $W_{16}$ & 9.792 & 0.002 & $H_0 \text{ ditolak}$ \\
            \cline{1-1}
            \multirow{4}{*}{Yuridiksi} & $W_{21}$ & 0.012 & 0.912 & $H_0 \text{ gagal ditolak}$ \\
            & $W_{22}$ & 0.139 & 0.710 & $H_0 \text{ gagal ditolak}$ \\
            & $W_{24}$ & 12.464 & 0.000 & $H_0 \text{ ditolak}$ \\
            & $W_{26}$ & 0.251 & 0.616 & $H_0 \text{ gagal ditolak}$ \\
            \hline
        \end{tabular}
    
        \item[-] Kesimpulan \\
        Pada taraf signifikansi 5\%, $\beta_{14}, \beta_{16}, \beta_{24}$ menghasilkan $H_0$ ditolak, sedangkan $\beta_{11}, \beta_{12}, \beta_{21}, \beta_{22}, \beta_{26}$ menghasilkan $H_0$ tidak ditolak. Akan tetapi karena variabel Lama Kasus pada Jenis Putusan Ajudikasi merupakan 
        satu kesatuan dengan Lama Kasus pada Jenis Putusan Mediasi yang bernilai signifikan maka disimpulkan variabel LK signifikan dibarengi variabel Jenis Pemohon juga signifikan, sementara lainnya variabel bersifat tidak signifikan.
    \end{enumerate}
    }
    \end{test}
    
    \item Uji Kesesuaian Model
    \begin{test}{
    \begin{enumerate}
        \item[-] Rumusan Hipotesis \\
        \begin{fleqn}[\parindent]
            \begin{equation*}
            \begin{split}
            &H_0 : \text{Model sesuai (observasi dan prediksi tidak berbeda)} \\ 
            &H_1 : \text{Model tidak sesuai (observasi dan prediksi berbeda)} \\
            \end{split}
            \end{equation*}
        \end{fleqn}
    
        \item[-] Taraf Signifikansi \\
        $\alpha = 0.05$
    
        \item[-] Statistik Uji \\
        Pada Tabel Hosmer and Lemeshow Test, dapat diperoleh : \\
        C = 12.118, Nilai sig = 0.587 \\
        Berdasarkan tabel F didapat : \\
        $\chi$-tabel = $\chi_{\alpha, k}^2$ = $\chi_{0.05; 14}$ = 23.685
    
        \item[-] Daerah Kritis \\
        $H_0$ ditolak jika $C > \chi_{\alpha, k}^2$ atau nilai sig $< \alpha$
    
        \item[-] Keputusan \\
        $H_0$ gagal ditolak karena $GC < \chi_{\alpha, k}^2$, yaitu $12.118 < 23.685$
    
        \item[-] Kesimpulan \\
        Pada taraf signifikansi 5\%, $H_0$ gagal ditolak sehingga model sesuai (observasi dan prediksi tidak berbeda).
    \end{enumerate}
    }
    \end{test}
    
    \item Odds Ratio
    \begin{test}{
        Pada Tabel Parameter Estimates, dapat diperoleh : \\
        $e^{\beta_{11}}$ = 0.942, $e^{\beta_{12}}$ = 0.943, $e^{\beta_{14}}$ = 0.326, $e^{\beta_{16}}$ = 0.361, $e^{\beta_{21}}$ = 0.949, $e^{\beta_{22}}$ = 0.848, $e^{\beta_{24}}$ = 0.338, $e^{\beta_{26}}$ = 1.165, 
    }
    \end{test}
    
    \item Keakuraten Model
    \begin{test}{
        Pada Tabel Classification, dapat diperoleh : \\
        hit ratio = 50.0\%
    }
    \end{test}

\end{enumerate} 
